% Partie 2: thème arithmétique

\section{Thème arithmétique}

    \subsection{Code source}

        \lstinputlisting{\SRCPATH/arithmetique.pl}

	Le prédicat element renvoie le n-ième élément d'une liste donnée en
	entrée ou donne les indices auxquels apparaît un élément donné dans
	une liste.

	L'appel est le suivant : \textbf{element( ?N, ?X, +L )}, où :
	\begin {itemize}
	    \item N est l'indice
	    \item X est l'élément
	    \item L est la liste
	\end {itemize}

    \subsection{Test}
	On teste ici chaque utilisation du prédicat element.

	Récupérer le n-ième élément d'une liste :

	\begin {lstlisting}
	    element( 7, X, [a,b,c,d,a,q,r,a,r]).
	    X = r ;
	    
	    element( 42, X, [a,b,c,d,a,q,r,a,r]).
	    false.
	\end {lstlisting}

	Récupérer les indices auquels un élément donné apparaît dans une liste
	:

	\begin {lstlisting}
	    element( N, a, [a,b,c,d,a,q,r,a,r]).
	    N = 1 ;
	    N = 5 ;
	    N = 8 ;
	    
	    element( N, z, [a,b,c,d,a,q,r,a,r]).
	    true .
	\end {lstlisting}

    \subsection{Bilan}


    Le prédicat ne fonctionne pas de manière optimale ; en effet, lorsqu'il
    est utilisé pour trouver les indices d'un élément donné, il devrait
    renvoyer false si l'élément en question ne se trouve pas dans la liste.
