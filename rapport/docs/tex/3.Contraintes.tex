% Partie 3: thème contraintes 

\section{Thème des contraintes}

    \subsection{Code source}

        \lstinputlisting{\SRCPATH/zebre.pl}

            On donne un numéro à chaque maison, puis on pose les 
contraintes données par l'énoncé. Il ne reste plus qu'à faire la 
corespondance entre le numéro de la maison et la propriété recherchée.    

    \subsection{Test}

L'exécution du programme dans {\ttfamily{swipl}} donne:

\begin{verbatim}
?- zebre(Nationalites, Boissons, Animaux, Cigarettes, Couleurs).
Nationalites = [1, 3, 5, 2, 4],
Boissons = [3, 4, 2, 5, 1],
Animaux = [4, 5, 3, 1, 2],
Cigarettes = [1, 4, 3, 2, 5],
Couleurs = [2, 3, 1, 4, 5].

?-
\end{verbatim}

   Ainsi, 
\begin{description}
    \item la maison 1 est jaune, habitée par un norvégien qui boit de 
l'eau, fume des Kools et possède un renard;

    \item la mainson 2 est bleue, habitée par un ukrainien qui boit du 
the, fume des Chesterfields et possède un cheval;

    \item la maison 3 est rouge, habitée par un anglais  qui boit du 
lait, fume des Oldgolds et possède un escargot;

    \item la maison 4 est verte, habitée par un japonais qui boit du 
café, fume des Cravens et possède un zèbre;

    \item et enfin la mainson 5 est blanche, habitée par un espagnol qui
boit du vin, fume des Gitanes et possède un chien.
\end{description}

    \subsection{Bilan}

        Cet exercice nous a permis de nous familiariser avec la 
programmation par contrainte, qui nécessite de bien poser le problème
pour laisser Prolog le résoudre.
