% Partie 1: thème généalogie.

\section{Thème généalogie}

    \subsection{Code source}

        \lstinputlisting{\SRCPATH/genealogie.pl}

            Nous avons créé 3 relations: {\ttfamily m(homme)} et {\ttfamily f(femme)} pour 
le sexe, et {\ttfamily parent(parent, enfant)} pour lier un parent à un enfant.

Cela nous permet de créer des prédicats comme {\ttfamily pere(pere, enfant)} ou 
{\ttfamily mere(mere, enfat)}, {\ttfamily frere(personne1, personne2)} ou encore 
{\ttfamily cousin(personne1, personne2)}.

Ces prédicats permettent d'exprimer des requêtes sur l'arbre généalogique.

    \subsection{Test}

Différentes requêtes:
\begin{description}
    \item Les parents de Jean:\\
\begin{verbatim}
?- parent(Parent,jean).
Parent = renee ;
Parent = brice ;
false.

?-
\end{verbatim}
    \item Les enfants de Marie:\\
\begin{verbatim}
?- parent(marie,Enfant).
Enfant = marc ;
Enfant = martin ;
Enfant = murielle.

?- 
\end{verbatim}
    \item Les frères et soeurs de Paul:\\
\begin{verbatim}
?- sibling(paul,Sibling).
Sibling = jean ;
Sibling = octave ;
Sibling = cosinus.

?-
\end{verbatim}
    \item Paul et Octave sont-ils de la même fratrie?\\
\begin{verbatim}
?- sibling(paul,octave).
true .

?-
\end{verbatim}
    \item Qui sont les cousins de Josh?\\
\begin{verbatim}
?- cousin(josh, Cousin).
Cousin = obiwan ;
Cousin = toto ;
false.

?-
\end{verbatim}
    \item Qui sont les femmes?\\  
\begin{verbatim}
?- f(Femme).
Femme = yoko ;
Femme = ursula ;
Femme = marie ;
Femme = renee ;
Femme = murielle.

?-
\end{verbatim}
\end{description}    
... et ainsi de suite.
    \subsection{Bilan}
        Cet exercice nous a permis d'utiliser Prolog comme une base de
données et d'effectuer des requêtes dessus.

